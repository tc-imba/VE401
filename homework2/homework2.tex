\documentclass[11pt,a4paper]{article}

\usepackage{../ve401}

\author{Group 37}
\semester{Spring}
\year{2019}
\subtitle{Assignment}
\subtitlenumber{2}
\blockinfo{
	\bigskip
	\begin{center}
		\textbf{Group members}
	\end{center}
	\begin{itemize}\itemsep .25cm
		\item \href{mailto:hcm_9809@sjtu.edu.cn}{Chenmin Hou} (517370910248)
		\item \href{mailto:liuyh615@126.com}{Yihao Liu} (515370910207)
		\item \href{mailto:lyz0123@sjtu.edu.cn}{Yuzhou Li} (517021910922)
	\end{itemize}
}

\begin{document}

\maketitle

\subsection{Drawing until First Success in the Hypergeometric Setting}

\begin{enumerate}[label=\roman*)]
\item
Let $H[n]$ be the unit step function, such that 
$$H[n]=\left\{\begin{aligned}
0,&\quad n<0\\1,&\quad n\geqslant 0
\end{aligned}\right..$$
Since ${\rm ran}\ X\subset \mathbb{N}$,
$$E[X]=\sum_{x\in N}x\cdot f_X(x)=\sum_{x=1}^\infty x\cdot P[X=x]=\sum_{x=0}^\infty x\cdot P[X=x].$$
And
\begin{align*}
\sum_{x=0}^\infty P[X>x]
&=\sum_{x=0}^\infty\sum_{y=x+1}^\infty P[X=y]\\
&=\sum_{x=0}^\infty\sum_{y=0}^\infty P[X=y]H[y-x-1]\\
&=\sum_{y=0}^\infty P[X=y]\sum_{x=0}^\infty H[y-x-1]\\
&=\sum_{y=0}^\infty P[X=y]\cdot y\\
&=E[X].
\end{align*}
\item
Let $x$ be the number of balls drawn, $x\in [0,N-r+1]\cap\mathbb{N}$
$$P[X>x]=\frac{\binom{r}{0}\binom{N-r}{x-0}}{\binom{N}{x}}=\frac{\binom{N-r}{x}}{\binom{N}{x}}=\frac{\frac{(N-r)!}{x!(N-r-x)!}}{\frac{N!}{x!(N-x)!}}=\frac{(N-r)!(N-x)!}{N!(N-r-x)!}.$$
According to i),
$$E[X]=\sum_{x=0}^\infty P[X>x]=\frac{(N-r)!}{N!}\sum_{x=0}^{N-r} \frac{(N-x)!}{(N-r-x)!}=\frac{(N-r)!r!}{N!}\sum_{x=0}^{N-r}\binom{N-x}{r}.$$
Let $k=N-r-x$, $b=N-r$,
\begin{align*}
E[X]&=\frac{(N-r)!r!}{N!}\sum_{k=0}^{N-r}\binom{r+k}{r}
=\frac{(N-r)!r!}{N!}\sum_{k=0}^{b}\binom{N-b+k}{N-b}\\
&=\frac{(N-r)!r!}{N!}\binom{N+1}{N-b+1}
=\frac{(N-r)!r!}{N!}\cdot\frac{(N+1)!}{(r+1)!(N-r)!}\\
&=\frac{N+1}{r+1}.
\end{align*}
\end{enumerate}

\subsection{Density of the Poisson Approximation}

Suppose $$p_x(t)=\frac{(\lambda t)^xe^{-\lambda t}}{x!}.$$
When $x=0$,
$$p_0'=-\lambda p_0\Longrightarrow p_0(t)=Ce^{-\lambda t}.$$
Since the probability of $x=0$ arrivals at time $t=0$ is 1, we can obtain $p_0(0)=1$, then
$$p_0(t)=e^{-\lambda t},$$
which satisfy the formula of $p_x(t)$.

When $x=x+1$,
$$p_{x+1}'(t)+\lambda p_{x+1}(t)=\lambda p_x(t)=\frac{\lambda(\lambda t)^xe^{-\lambda t}}{x!}.$$
Let $p_{x+1}(t)=u(t)e^{-\int\lambda dt}=u(t)e^{-\lambda t}$,
$$u'(t)=\frac{\frac{\lambda(\lambda t)^xe^{-\lambda t}}{x!}}{e^{-\int\lambda dt}}=\frac{\lambda(\lambda t)^x}{x!}.$$
$$p_{x+1}(t)=e^{-\lambda t}\int u'(t)dt=e^{-\lambda t}\int\frac{\lambda(\lambda t)^x}{x!}dt=e^{-\lambda t}\frac{1}{(x+1)\lambda}\frac{\lambda(\lambda t)^{x+1}}{x!}=\frac{(\lambda t)^{x+1}e^{-\lambda t}}{(x+1)!},$$
which satisfy the formula of $p_x(t)$.

So it is proved by induction.

\end{document}



\documentclass[11pt,a4paper]{article}

\usepackage{../ve401}

\author{Group 37}
\semester{Spring}
\year{2019}
\subtitle{Assignment}
\subtitlenumber{3}
\blockinfo{
	\bigskip
	\begin{center}
		\textbf{Group members}
	\end{center}
	\begin{itemize}\itemsep .25cm
		\item \href{mailto:hcm_9809@sjtu.edu.cn}{Chenmin Hou} (517370910248)
		\item \href{mailto:liuyh615@126.com}{Yihao Liu} (515370910207)
		\item \href{mailto:lyz0123@sjtu.edu.cn}{Yuzhou Li} (517021910922)
	\end{itemize}
}

\begin{document}

\maketitle

\subsection{}

\begin{enumerate}[label=\roman*)]
\item
From the definition of $f_{XY}(x,y)$, we know $f_{XY}(x,y)\geqslant0$ for all $(x,y)\in\Omega$, and
$$\sum_{(x,y)\in\Omega}f_{XY}(x,y)=\sum_{x=1}^n\sum_{y=x}^n\frac{2}{n(n+1)}=\sum_{x=1}^nx\frac{2}{n(n+1)}=\frac{n(n+1)}{2}\cdot\frac{2}{n(n+1)}=1.$$
So $f_{XY}$ is in fact a density.
\item
\begin{align*}
f_X(x)&=\sum_yf_{XY}(x,y)=\sum_{y=x}^n\frac{2}{n(n+1)}=\frac{2(n+1-x)}{n(n+1)},\\
f_Y(y)&=\sum_xf_{XY}(x,y)=\sum_{x=1}^y\frac{2}{n(n+1)}=\frac{2y}{n(n+1)}.
\end{align*}
\item
$$f_X(x)f_Y(y)=\frac{4(n+1-x)y}{n^2(n+1)^2},$$
$$f_{XY}(n,1)=\frac{2}{n(n+1)}\neq f_X(n)f_Y(1)=\frac{4}{n^2(n+1)^2}.$$
So $X$ and $Y$ are not independent.
\item
\begin{align*}
P[X\leqslant3{\rm\ and\ }Y\leqslant2]&=\sum_{x=1}^3\sum_{y=x}^2f_{XY}(x,y)=\sum_{x=1}^3\sum_{y=x}^2\frac{2}{n(n+1)}=3\cdot\frac{2}{5\cdot(5+1)}=\frac{1}{5},\\
P[X\leqslant3]&=\sum_{x=1}^3f_X(x)=\sum_{x=1}^3\frac{2(n+1-x)}{n(n+1)}=(5+4+3)\cdot\frac{2}{5\cdot(5+1)}=\frac{4}{5},\\
P[Y\leqslant2]&=\sum_{y=1}^2f_Y(y)=\sum_{y=1}^2\frac{2y}{n(n+1)}=(1+2)\cdot\frac{2}{5\cdot(5+1)}=\frac{1}{5}.
\end{align*}
\end{enumerate}

\subsection{The Sum of Two Continuous Random Variables}

Consider the transformation 
$$\varphi(X,Y)\mapsto (U,V)$$
where
$$\varphi(x,y)=\binom{x+y}{y}.$$
Then
$$\varphi^{-1}(u,v)=\binom{u-v}{v}.$$
We calculate
$$D\varphi^{-1}(u,v)=\begin{pmatrix}
\frac{\partial x}{\partial u} & \frac{\partial x}{\partial v} \\
\frac{\partial y}{\partial u} & \frac{\partial y}{\partial v} \\
\end{pmatrix}=\begin{pmatrix}
1 & -1 \\ 0 & 1 \\
\end{pmatrix},$$
so
$$\left|\frac{\partial(x,y)}{\partial(u,v)}\right||\det D\varphi^{-1}(u,v)|=1.$$
Then
$$f_{UV}(u,v)=f_{XY}(x,y)|\det D\varphi^{-1}(u,v)|=f_{XY}(x,y).$$
The marginal density $f_U$ is given by
$$f_U(u)=\int_{-\infty}^\infty f_{UV}(u,v)dv=\int_{-\infty}^\infty f_{XY}(x,y)dy=\int_{-\infty}^\infty f_{XY}(u-v,v)dv.$$

\subsection{The Sum of Two Exponential Distributions}
Let
$$f_X(x)=\left\{\begin{aligned}
&\beta_1e^{-\beta_1x}=\frac{1}{3}e^{-x/3} &\quad x>0 \\
&0 &\quad x\leqslant 0
\end{aligned}\right.,\quad
f_Y(y)=\left\{\begin{aligned}
&\beta_2e^{-\beta_2y}=e^{-y} &\quad x>0 \\
&0 &\quad x\leqslant 0
\end{aligned}\right.,$$
and
$$f_{XY}(x,y)=f_X(x)f_Y(y)=\left\{\begin{aligned}
&\frac{1}{3}e^{-x/3-y} &\quad x>0 \\
&0 &\quad x\leqslant 0
\end{aligned}\right..$$
According to Exercise 3.2,
$$f_U(u)=\int_{-\infty}^\infty f_{XY}(u-v,v)dv.$$
When $u,v>0$, $u-v>0$, we can get $0<v<u$, then
$$f_U(u)=\int_0^u\frac{1}{3}e^{-(u-v)/3-v}dv=\frac{1}{3}e^{-u/3}\int_0^u e^{-2v/3}dv=\frac{1}{3}e^{-u/3}\cdot-\frac{3}{2}e^{-2v/3}\bigg|_0^u=\frac{1}{2}(e^{-u/3}-e^{-u}).$$
When $u,v\leqslant0$, $f_U(u)=0$, so
$$f_U(u)=\left\{\begin{aligned}
&\frac{1}{2}(e^{-u/3}-e^{-u}) &\quad u>0 \\
&0 &\quad u\leqslant 0
\end{aligned}\right..$$
\end{document}



\documentclass[11pt,a4paper]{article}

\usepackage{../ve401}
\usepackage{graphicx}

\author{Group 37}
\semester{Spring}
\year{2019}
\subtitle{Assignment}
\subtitlenumber{4}
\blockinfo{
	\bigskip
	\begin{center}
		\textbf{Group members}
	\end{center}
	\begin{itemize}\itemsep .25cm
		\item \href{mailto:hcm_9809@sjtu.edu.cn}{Chenmin Hou} (517370910248)
		\item \href{mailto:liuyh615@126.com}{Yihao Liu} (515370910207)
		\item \href{mailto:lyz0123@sjtu.edu.cn}{Yuzhou Li} (517021910922)
	\end{itemize}
}

\begin{document}

\maketitle

\subsection{}
$$1-\alpha=0.95\Longrightarrow a=0.05.$$
A 95\% two sided confidence interval for $\sigma^2$ is
$$\left[\frac{(n-1)s^2}{\chi^2_{\alpha/2,n-1}},\frac{(n-1)s^2}{\chi^2_{1-\alpha/2,n-1}}\right]=\left[\frac{50\cdot0.37^2}{71.420},\frac{50\cdot0.37^2}{32.357}\right]\approx0.09584,0.2115].$$
So a 95\% two sided confidence interval for $\sigma$ is 
$$[\sqrt{0.09584},\sqrt{0.2115}]\approx[0.31,0.46].$$

\subsection{Non-Parametric Confidence Intervals}

\begin{enumerate}[label=\roman*)]
\item
The median of a random variable $X$ is defined as
$$P[X>M]=P[X<M]=\frac{1}{2}.$$
Then
\begin{align*}
&P[X_{min}\leqslant M\leqslant X_{max}]\\
=&1-P[X_{min}>M]-P[X_{max}<M]\\
=&1-\prod_{i=1}^nP[X>M]-\prod_{i=1}^nP[X<M]\\
=&1-\left(\frac{1}{2}\right)^{n-1}.
\end{align*}
\item
\begin{align*}
&P[X_{k+1}\leqslant M\leqslant X_{n-k}]\\
=&1-P[X_{k+1}>M]-P[X_{n-k}<M]\\
=&1-\sum_{i=0}^k\binom{n}{k}P[X_i\leqslant M]P[X_{i+1}>M]-\sum_{i=0}^k\binom{n}{k}P[X_{n-i+1}\geqslant M]P[X_{n-i}<M]\\
=&1-\sum_{i=0}^k\binom{n}{k}\prod_{j=1}^iP[X\leqslant M]\prod_{j=i+1}^nP[X>M]-\sum_{i=0}^k\binom{n}{k}\prod_{j=n-i+1}^nP[X\geqslant M]\prod_{j=1}^{n-i}P[X<M]\\
=&1-2\sum_{i=0}^k\binom{n}{k}\left(\frac{1}{2}\right)^n.
\end{align*}
\end{enumerate}

\subsection{Fisher Test}
We set up the null hypothesis $$H_0:\mu\leqslant25.$$
Since there are only 20 samples, we should use the T-Test. According to the data, we obtain
$$\overline{X}=26.035,\quad S=4.785.$$
Therefore, the test statistic is
$$T_{19}=\frac{\overline{X}-\mu_0}{S/\sqrt{n}}=\frac{26.035-25}{4.785/\sqrt{20}}\approx0.967,$$
Then $$P[T_{19}\geqslant0.967\mid H_0]=1-P[T_{19}\leqslant0.967\mid H_0]=1-0.8272=0.1728.$$
Since the P-value is not small enough, I can't support the claim.

\subsection{Neyman-Pearson Decision Test}

\begin{enumerate}[label=\roman*)]
\item
\item
\item
\item
\end{enumerate}

\subsection{}

\end{document}


\documentclass[11pt,a4paper]{article}

\usepackage{../ve401}
\usepackage{graphicx}

\author{Group 37}
\semester{Spring}
\year{2019}
\subtitle{Assignment}
\subtitlenumber{5}
\blockinfo{
	\bigskip
	\begin{center}
		\textbf{Group members}
	\end{center}
	\begin{itemize}\itemsep .25cm
		\item \href{mailto:hcm_9809@sjtu.edu.cn}{Chenmin Hou} (517370910248)
		\item \href{mailto:liuyh615@126.com}{Yihao Liu} (515370910207)
		\item \href{mailto:lyz0123@sjtu.edu.cn}{Yuzhou Li} (517021910922)
	\end{itemize}
}

\begin{document}

\maketitle

\subsection{}

\subsection{}

\subsection{}

\subsection{}

\begin{enumerate}[label=\roman*)]
\item
According to Theorem 2.9.1, since $X_1$ and $X_2$ are samples of two different solid-fuel propellants, they are independent. By Central Limit Theorem we can approximate that they follow normal distributions with mean $\mu_1$ and $\mu_2$ and variance $\sigma_1^2$ and $\sigma_2^2$, then $\overline{X_1}-\overline{X_2}$ follows a normal distribution
$$Z=\frac{(\overline{X_1}-\overline{X_2})-(\mu_1-\mu_2)}{\sqrt{\sigma_1^2/n_1+\sigma_2^2/n_2}},$$ 
where $n_1$ and $n_2$ are the sample sizes, and $\overline{X_1}$ and $\overline{X_2}$ are the sample means of the two materials. The mean is $\mu_1-\mu_2$ and the variance is $\sqrt{\sigma_1^2/n_1+\sigma_2^2/n_2}=9/10$, and the equivalent sample size $n$ is 10. A 95\% confidence interval on the difference in means $\mu_1-\mu_2$ is
$$(\overline{X_1}-\overline{X_2})\pm\frac{z_{\alpha/2}\cdot \sigma}{\sqrt{n}}=-6\pm\frac{1.96\cdot\sqrt{9/10}}{\sqrt{10}}=[-6.588,-5.412].$$
The practical meaning of this interval is that based on the samples, we are 95\% certain that the the difference in means lies in $[-6.588,-5.412]$.
\item
This is a two tailed test, so using $\alpha=0.05$, we look up $z_{\alpha/2}=1.96$. The value of the test statistic is
$$Z_0=\frac{\overline{x_1}-\overline{x_2}}{\sqrt{\sigma_1^2/n_1+\sigma_2^2/n_2}}=\frac{-6}{\sqrt{9/10}}\approx-6.32.$$
Since $|Z_0|>1.96$, we reject $H_0$ at a 5\% level of significance.
\item
Using $\beta=0.1$, we look up $z_\beta=1.28$. According to Theorem 2.5.4, 
$$n\approx\frac{(z_{\alpha/2}+z_\beta)^2\sigma^2}{\delta^2}=\frac{(1.96+1.27)^2\cdot9/10}{14^2}\approx0.048.$$
So the sample size should be at least 1 to obtain a power of 0.9 in this situation.
\end{enumerate}


\end{document}


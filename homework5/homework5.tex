\documentclass[11pt,a4paper]{article}

\usepackage{../ve401}
\usepackage{graphicx}

\author{Group 37}
\semester{Spring}
\year{2019}
\subtitle{Assignment}
\subtitlenumber{5}
\blockinfo{
	\bigskip
	\begin{center}
		\textbf{Group members}
	\end{center}
	\begin{itemize}\itemsep .25cm
		\item \href{mailto:hcm_9809@sjtu.edu.cn}{Chenmin Hou} (517370910248)
		\item \href{mailto:liuyh615@126.com}{Yihao Liu} (515370910207)
		\item \href{mailto:lyz0123@sjtu.edu.cn}{Yuzhou Li} (517021910922)
	\end{itemize}
}

\begin{document}

\maketitle

\subsection{}

\subsection{}

\subsection{}

\subsection{}

\begin{enumerate}[label=\roman*)]
\item
According to Theorem 2.9.1, since $X_1$ and $X_2$ are samples of two different solid-fuel propellants, they are independent. By Central Limit Theorem we can approximate that they follow normal distributions with mean $\mu_1$ and $\mu_2$ and variance $\sigma_1^2$ and $\sigma_2^2$, then $\overline{X_1}-\overline{X_2}$ follows a normal distribution
$$Z=\frac{(\overline{X_1}-\overline{X_2})-(\mu_1-\mu_2)}{\sqrt{\sigma_1^2/n_1+\sigma_2^2/n_2}},$$ 
where $n_1$ and $n_2$ are the sample sizes, and $\overline{X_1}$ and $\overline{X_2}$ are the sample means of the two materials. The mean is $\mu_1-\mu_2$ and the variance is $\sqrt{\sigma_1^2/n_1+\sigma_2^2/n_2}=9/10$, and the equivalent sample size $n$ is 10. A 95\% confidence interval on the difference in means $\mu_1-\mu_2$ is
$$(\overline{X_1}-\overline{X_2})\pm\frac{z_{\alpha/2}\cdot \sigma}{\sqrt{n}}=-6\pm\frac{1.96\cdot\sqrt{9/10}}{\sqrt{10}}=[-6.588,-5.412].$$
The practical meaning of this interval is that based on the samples, we are 95\% certain that the the difference in means lies in $[-6.588,-5.412]$.
\item
This is a two tailed test, so using $\alpha=0.05$, we look up $z_{\alpha/2}=1.96$. The value of the test statistic is
$$Z_0=\frac{\overline{x_1}-\overline{x_2}}{\sqrt{\sigma_1^2/n_1+\sigma_2^2/n_2}}=\frac{-6}{\sqrt{9/10}}\approx-6.32.$$
Since $|Z_0|>1.96$, we reject $H_0$ at a 5\% level of significance.
\item
Using $\beta=0.1$, we look up $z_\beta=1.28$. Similar to Theorem 2.5.4, the relationship between $\delta$, $\alpha$, $\beta$ and $n$ with $\sigma_1$, $\sigma_2$ is given by
$$-z_\beta\approx z_{\alpha/2}-\delta/\sqrt{\sigma_1^2/n+\sigma_2^2/n}=z_{\alpha/2}-\delta\sqrt{n}/\sqrt{\sigma_1^2+\sigma_2^2},$$
$$n\approx\frac{(z_{\alpha/2}+z_\beta)^2(\sigma_1^2+\sigma_2^2)}{\delta^2}=\frac{(1.96+1.27)^2\cdot(9+9)}{14^2}\approx0.958.$$
So the sample size should be at least 1 to obtain a power of 0.9 in this situation.
\end{enumerate}

\subsection{}
%According to Theorem 2.9.7, since sample variances $S_1^2$ and $S_2^2$ are based on independent random samples of sizes $n_1$ and $n_2$, then the statistic $$F=\frac{S_2^2/\sigma_2^2}{S_1^2/\sigma_1^2}$$ follows an F-distribution with $n_2-1$ and $n_1-1$ degrees of freedom. Therefore
%$$P[f_{1-\alpha/2,n_2-1,n_1-1}\leqslant F\leqslant f_{\alpha/2,n_2-1,n_1-2}]=1-\alpha,$$
%then an $100(1-\alpha)\%$ confidence interval on the ratio $\sigma_1^2/\sigma_2^2$ is
%$$\frac{s_1^2}{s_2^2}f_{1-\alpha/2,n_2-1,n_1-1}\leqslant\frac{\sigma_1^2}{\sigma_2^2}\leqslant \frac{s_1^2}{s_2^2}f_{\alpha/2,n_2-1,n_1-2}.$$
%When $n_1=10$, $s_1=4.7$, $n_2=16$, $s_2=5.8$, $\alpha=0.05$, we can obtain
%$$\frac{s_1^2}{s_2^2}f_{0.975,15,9}\leqslant\frac{\sigma_1^2}{\sigma_2^2}\leqslant\frac{s_1^2}{s_2^2}f_{0.025,15,9},$$
%$$\frac{4.7^2}{5.8^2}\cdot0.32\leqslant\frac{\sigma_1^2}{\sigma_2^2}\leqslant\frac{4.7^2}{5.8^2}\cdot3.77$$
%$$0.458\leqslant\frac{\sigma_1}{\sigma_2}\leqslant1.573$$

Suppose that we wish to test
$$H_0:\sigma_1^2=\sigma_2^2\quad,\quad H_1:\sigma_1^2\neq\sigma_2^2$$
Since $n_1=10$, $s_1=4.7$, $n_2=16$, $s_2=5.8$, $\alpha=0.05$, and
$$f_0=\frac{s_1^2}{s_2^2}=\frac{4.7^2}{5.8^2}\approx0.657.$$
We will reject $H_0$ if $f_0>f_{\alpha/2,n_1-1,n_2-1}=f_{0.025,9,15}=3.12$ or $f_0<f_{1-\alpha/2,n_1-1,n_2-1}=f_{0.975,9,15}=0.27$. Since 
$$0.27<f_0<3.12,$$
we cannot reject $H_0$, there isn't sufficient evidence to conclude that the two population variances differ by at least 0.5 grams per liter.

\subsection{}
Suppose that we wish to test
$$H_0:\sigma_1^2=\sigma_2^2\quad,\quad H_1:\sigma_1^2\neq\sigma_2^2$$
According to the data, $n_1=10$, $s_1^2=1.477\times10^{-3}$, $n_2=12$, $s_2^2=1.196\times10^{-2}$, and 
$$f_0=\frac{s_1^2}{s_2^2}=\frac{1.477\times10^{-3}}{1.196\times10^{-2}}\approx0.123.$$
We will reject $H_0$ if $f_0>f_{\alpha/2,n_1-1,n_2-1}=f_{\alpha/2,9,11}$ or $f_0<f_{1-\alpha/2,n_1-1,n_2-1}=f_{1-\alpha/2,9,11}$. Since $f_0$ is relatively small, we only consider $f_0<f_{1-\alpha/2,9,11}$,
$$1-\alpha/2=P[F_{9,11}<f_{1-\alpha/2,9,11}],$$
When $\alpha<0.004$, $f_0>f_{1-\alpha/2,9,11}$, so the P-value of the test is 0.004. The conclusion is that we can reject $H_0$ at at significance level 0.4\%, there is very much evidence to reject $H_0$.


\end{document}


\documentclass[11pt,a4paper]{article}

\usepackage{../ve401}
\usepackage{graphicx}

\author{Group 37}
\semester{Spring}
\year{2019}
\subtitle{Assignment}
\subtitlenumber{5}
\blockinfo{
	\bigskip
	\begin{center}
		\textbf{Group members}
	\end{center}
	\begin{itemize}\itemsep .25cm
		\item \href{mailto:hcm_9809@sjtu.edu.cn}{Chenmin Hou} (517370910248)
		\item \href{mailto:liuyh615@126.com}{Yihao Liu} (515370910207)
		\item \href{mailto:lyz0123@sjtu.edu.cn}{Yuzhou Li} (517021910922)
	\end{itemize}
}

\begin{document}

\maketitle

\subsection{Neyman-Pearson Decision Test}

\begin{enumerate}[label=\roman*)]
\item
$$\frac{\overline{X}-\mu_0}{\sigma/\sqrt{n}}\geqslant z_{\alpha},$$
$$\frac{\overline{X}-4}{0.2/\sqrt{50}}\geqslant\Phi^{-1}(0.95)\approx1.645.$$
The critical region is determined by
$$\overline{X}>4.046.$$
\item
\begin{align*}
{\rm Power}&=P[\overline{X}\geqslant 4.046\mid \mu\geqslant 4.5,\sigma=0.2]\\
&\geqslant P[\overline{X}\geqslant 4.046\mid \mu= 4.5,\sigma=0.2]\\
&=P\left[\frac{\overline{X}-4.5}{0.2/\sqrt{50}}\geqslant\frac{4.046-4.5}{0.2/\sqrt{50}}\right]\\
&=P[Z\geqslant -16.05]=1-P[Z\leqslant -16.05]\\
&\approx 1-0=1.
\end{align*}
\item
$$1-\beta'\geqslant0.97\Longrightarrow\beta'<0.03,$$
$$z_{\beta'}=-\Phi^{-1}(\beta')\gtrapprox 1.88,$$
$$n\approx\frac{(z_\alpha+z_{\beta'})^2\sigma^2}{\delta^2}\gtrapprox\frac{(1.645+1.88)^2\cdot0.2^2}{0.5^2}=1.9881.$$
So the required sample size is 2.
\item
Since $\overline{x}>4.046$, $H_0$ is rejected.
$$\overline{x}-z_\alpha\frac{\sigma}{\sqrt{n}}=4.05-1.645\cdot\frac{0.2}{\sqrt{50}}\approx4.0035.$$
So a 95\% confidence interval for $\mu$ is $\mu\geqslant4.0035$.
\end{enumerate}

\subsection{NHST}

\begin{enumerate}[label=\roman*)]
\item
$$\frac{\overline{X}-\mu_0}{\sigma/\sqrt{n}}\geqslant z_{\alpha},$$
$$\frac{\overline{X}-100}{4/\sqrt{8}}\geqslant\Phi^{-1}(0.95)\approx1.645.$$
The critical region is determined by
$$\overline{X}>102.33.$$
Since $\bar{x}=102.2<102.33$, there is no evidence to reject $H_0$ at 5\% level of significance.
\item
\begin{align*}
&P[\overline{X}\geqslant 102.2\mid \mu\leqslant 100,\sigma=4]\\
\leqslant & P[\overline{X}\geqslant 102.2\mid \mu=100,\sigma=4]\\
=&P\left[\frac{\overline{X}-100}{4/\sqrt{8}}\geqslant\frac{102.2-100}{4/\sqrt{8}}\right]\\
=&P[Z\geqslant 1.556]=1-P[Z\leqslant1.556]\\
\approx&1-0.94=0.06.
\end{align*}
\item
\begin{align*}
{\rm Power}&=P[\overline{X}\geqslant 102.33\mid \mu\geqslant 105,\sigma=4]\\
&\geqslant P[\overline{X}\geqslant 102.33\mid \mu=105,\sigma=4]\\
&=P\left[\frac{\overline{X}-105}{4/\sqrt{8}}\geqslant\frac{102.33-105}{4/\sqrt{8}}\right]\\
&=P[Z\geqslant -1.888]=1-P[Z\leqslant -1.888]\\
&\approx 1-0.03=0.97.
\end{align*}
\item
$$1-\beta'\geqslant0.85\Longrightarrow\beta'<0.15,$$
$$z_{\beta'}=-\Phi^{-1}(\beta')\gtrapprox 1.035,$$
$$n\approx\frac{(z_\alpha+z_{\beta'})^2\sigma^2}{\delta^2}\gtrapprox\frac{(1.645+1.035)^2\cdot4^2}{5^2}\approx4.597.$$
So the required sample size is 5.
\item
$$\overline{x}-z_\alpha\frac{\sigma}{\sqrt{n}}=102.2-1.645\cdot\frac{4}{\sqrt{8}}\approx99.874.$$
So a 95\% confidence interval for $\mu$ is $\mu\geqslant99.874$, and since $99.874<100$, there is no evidence to reject $H_0$ at 5\% level of significance.
\end{enumerate}

\subsection{}
\begin{enumerate}[label=\roman*)]
\item
We are using Fisher’s Hypothesis Test, we test $$H_0:p_1\leqslant p_2,$$
where $p_1$ denotes the proportion of successful repairs of larger tears and $p_2$ denotes that of shorter tears. Estimators for $p_1$ and $p_2$ are $$\hat{p_1}=14/18=\frac{7}{9}\quad,\quad\hat{p_2}=22/30=\frac{11}{15}.$$
The pooled estimate for the common population proportion is
$$\hat{p}=\frac{n_1\hat{p_1}+n_2\hat{p_2}}{n_1+n_2}=\frac{14+22}{18+30}=\frac{3}{4}.$$
The observed value of the test statistic is
$$Z=\frac{\hat{p_1}-\hat{p_2}}{\sqrt{\hat{p}(1-\hat{p})\left(\frac{1}{n_1}+\frac{1}{n_2}\right)}}=\frac{\frac{7}{9}-\frac{11}{15}}{\sqrt{\frac{3}{4}}\cdot\frac{1}{4}\cdot\left(\frac{1}{18}+\frac{1}{30}\right)}=\frac{4\sqrt{15}}{45}\approx0.344.$$
From the standard normal table, we see that the probability of observing
this large or a larger value is 0.36317, so we shall not reject $H_0$. Therefore, there is no sufficient evidence that the success rate is greater for longer tears.
\item
$$(\hat{p_1}-\hat{p_2})-z_{\alpha}\sqrt{\frac{\hat{p_1}(1-\hat{p_1})}{n_1}-\frac{\hat{p_2}(1-\hat{p_2})}{n_2}}=\left(\frac{7}{9}-\frac{11}{15}\right)-1.645\cdot\sqrt{\frac{\frac{7}{9}\cdot\frac{2}{9}}{18}+\frac{\frac{11}{15}\cdot\frac{4}{15}}{30}}\approx-0.1533.$$
So a 95\% lower confidence bound for $p_1-p_2$ is $p_1-p_2\geqslant -0.1533$, and since $-0.1533<0$, there is no sufficient evidence that the success rate is greater for longer tears.
\end{enumerate}

\subsection{}

\begin{enumerate}[label=\roman*)]
\item
According to Theorem 2.9.1, since $X_1$ and $X_2$ are samples of two different solid-fuel propellants, they are independent. By Central Limit Theorem we can approximate that they follow normal distributions with mean $\mu_1$ and $\mu_2$ and variance $\sigma_1^2$ and $\sigma_2^2$, then $\overline{X_1}-\overline{X_2}$ follows a normal distribution
$$Z=\frac{(\overline{X_1}-\overline{X_2})-(\mu_1-\mu_2)}{\sqrt{\sigma_1^2/n_1+\sigma_2^2/n_2}},$$ 
where $n_1$ and $n_2$ are the sample sizes, and $\overline{X_1}$ and $\overline{X_2}$ are the sample means of the two materials. The mean is $\mu_1-\mu_2$ and the variance is $\sqrt{\sigma_1^2/n_1+\sigma_2^2/n_2}=9/10$, and the equivalent sample size $n$ is 10. A 95\% confidence interval on the difference in means $\mu_1-\mu_2$ is
$$(\overline{X_1}-\overline{X_2})\pm\frac{z_{\alpha/2}\cdot \sigma}{\sqrt{n}}=-6\pm\frac{1.96\cdot\sqrt{9/10}}{\sqrt{10}}=[-6.588,-5.412].$$
The practical meaning of this interval is that based on the samples, we are 95\% certain that the the difference in means lies in $[-6.588,-5.412]$.
\item
This is a two tailed test, so using $\alpha=0.05$, we look up $z_{\alpha/2}=1.96$. The value of the test statistic is
$$Z_0=\frac{\overline{x_1}-\overline{x_2}}{\sqrt{\sigma_1^2/n_1+\sigma_2^2/n_2}}=\frac{-6}{\sqrt{9/10}}\approx-6.32.$$
Since $|Z_0|>1.96$, we reject $H_0$ at a 5\% level of significance.
\item
Using $\beta=0.1$, we look up $z_\beta=1.28$. Similar to Theorem 2.5.4, the relationship between $\delta$, $\alpha$, $\beta$ and $n$ with $\sigma_1$, $\sigma_2$ is given by
$$-z_\beta\approx z_{\alpha/2}-\delta/\sqrt{\sigma_1^2/n+\sigma_2^2/n}=z_{\alpha/2}-\delta\sqrt{n}/\sqrt{\sigma_1^2+\sigma_2^2},$$
$$n\approx\frac{(z_{\alpha/2}+z_\beta)^2(\sigma_1^2+\sigma_2^2)}{\delta^2}=\frac{(1.96+1.27)^2\cdot(9+9)}{14^2}\approx0.958.$$
So the sample size should be at least 1 to obtain a power of 0.9 in this situation.
\end{enumerate}

\subsection{}
%According to Theorem 2.9.7, since sample variances $S_1^2$ and $S_2^2$ are based on independent random samples of sizes $n_1$ and $n_2$, then the statistic $$F=\frac{S_2^2/\sigma_2^2}{S_1^2/\sigma_1^2}$$ follows an F-distribution with $n_2-1$ and $n_1-1$ degrees of freedom. Therefore
%$$P[f_{1-\alpha/2,n_2-1,n_1-1}\leqslant F\leqslant f_{\alpha/2,n_2-1,n_1-2}]=1-\alpha,$$
%then an $100(1-\alpha)\%$ confidence interval on the ratio $\sigma_1^2/\sigma_2^2$ is
%$$\frac{s_1^2}{s_2^2}f_{1-\alpha/2,n_2-1,n_1-1}\leqslant\frac{\sigma_1^2}{\sigma_2^2}\leqslant \frac{s_1^2}{s_2^2}f_{\alpha/2,n_2-1,n_1-2}.$$
%When $n_1=10$, $s_1=4.7$, $n_2=16$, $s_2=5.8$, $\alpha=0.05$, we can obtain
%$$\frac{s_1^2}{s_2^2}f_{0.975,15,9}\leqslant\frac{\sigma_1^2}{\sigma_2^2}\leqslant\frac{s_1^2}{s_2^2}f_{0.025,15,9},$$
%$$\frac{4.7^2}{5.8^2}\cdot0.32\leqslant\frac{\sigma_1^2}{\sigma_2^2}\leqslant\frac{4.7^2}{5.8^2}\cdot3.77$$
%$$0.458\leqslant\frac{\sigma_1}{\sigma_2}\leqslant1.573$$

Suppose that we wish to test
$$H_0:\sigma_1^2=\sigma_2^2\quad,\quad H_1:\sigma_1^2\neq\sigma_2^2$$
Since $n_1=10$, $s_1=4.7$, $n_2=16$, $s_2=5.8$, $\alpha=0.05$, and
$$f_0=\frac{s_1^2}{s_2^2}=\frac{4.7^2}{5.8^2}\approx0.657.$$
We will reject $H_0$ if $f_0>f_{\alpha/2,n_1-1,n_2-1}=f_{0.025,9,15}=3.12$ or $f_0<f_{1-\alpha/2,n_1-1,n_2-1}=f_{0.975,9,15}=0.27$. Since 
$$0.27<f_0<3.12,$$
we cannot reject $H_0$, there isn't sufficient evidence to conclude that the two population variances differ by at least 0.5 grams per liter.

\subsection{}
Suppose that we wish to test
$$H_0:\sigma_1^2=\sigma_2^2\quad,\quad H_1:\sigma_1^2\neq\sigma_2^2$$
According to the data, $n_1=10$, $s_1^2=1.477\times10^{-3}$, $n_2=12$, $s_2^2=1.196\times10^{-2}$, and 
$$f_0=\frac{s_1^2}{s_2^2}=\frac{1.477\times10^{-3}}{1.196\times10^{-2}}\approx0.123.$$
We will reject $H_0$ if $f_0>f_{\alpha/2,n_1-1,n_2-1}=f_{\alpha/2,9,11}$ or $f_0<f_{1-\alpha/2,n_1-1,n_2-1}=f_{1-\alpha/2,9,11}$. Since $f_0$ is relatively small, we only consider $f_0<f_{1-\alpha/2,9,11}$, and when $\alpha/2=0.002$, $f_{1-\alpha/2,9,11}\approx 0.123$.
$$\frac{\alpha}{2}=P[F_{9,11}<f_{1-\alpha/2,9,11}\mid \sigma_1^2=\sigma_2^2]< P[F_{9,11}<f_0\mid \sigma_1^2=\sigma_2^2]\approx0.002$$
Since it is a two-sided test, the P-value of the test is 0.004. The conclusion is that we can reject $H_0$ at at significance level 0.4\%, there is very much evidence to reject $H_0$.


\end{document}


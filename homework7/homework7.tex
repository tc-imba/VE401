\documentclass[11pt,a4paper]{article}

\usepackage{../ve401}
\usepackage{graphicx}

\author{Group 37}
\semester{Spring}
\year{2019}
\subtitle{Assignment}
\subtitlenumber{7}
\blockinfo{
	\bigskip
	\begin{center}
		\textbf{Group members}
	\end{center}
	\begin{itemize}\itemsep .25cm
		\item \href{mailto:hcm_9809@sjtu.edu.cn}{Chenmin Hou} (517370910248)
		\item \href{mailto:liuyh615@126.com}{Yihao Liu} (515370910207)
		\item \href{mailto:lyz0123@sjtu.edu.cn}{Yuzhou Li} (517021910922)
	\end{itemize}
}

\begin{document}

\maketitle

\subsection{}
First we let $D$ be the reaction time of auditory minus that of visual, and $M$ be the median of $D$. Then we have the value and signed rank of $D_i$ in the table:
\begin{center}
\small
\begin{tabular}{c|ccccccccccccccc}
Subject & 1 & 2 & 3 & 4 & 5 & 6 & 7 & 8 & 9 & 10 & 11 & 12 & 13 & 14 & 15 \\
Visual & 161 & 203 & 235 & 176 & 201 & 188 & 228 & 211 & 191 & 178 & 159 & 227 & 193 & 192 & 212 \\
Auditory & 157 & 207 & 198 & 161 & 234 & 197 & 180 & 165 & 202 & 193 & 173 & 137 & 182 & 159 & 156 \\
$D$ & -4 & +4 & -37 & -15 & +33 & +9 & -48 & -46 & +11 & +15 & +14 & -90 & -11 & -33 & -56 \\
Signed Rank & -1.5 & +1.5 & -11 & -7.5 & +9.5 & +3 & -13 & -12 & +4.5 & +7.5 & +6 & -15 & -4.5 & -9.5 & -14
\end{tabular}
\end{center}
\begin{enumerate}[label=\roman*)]
\item
With a paired T-test, we hence test $H_0 : \mu_D \leqslant 0$. From the $n = 15$ data we have $\overline{D} = -16.93$ and $s_D^2 = 1135.92$. The test statistic is
$$T=\frac{\overline{D}}{s_D/\sqrt{n}}=\frac{-16.93}{\sqrt{1135.92/15}}\approx-1.945.$$
Let $\alpha=0.05$, $t_{\alpha,15}=1.75$, and since $-1.945<-t_{\alpha,15}$, we can reject $H_0$ at 5\% level of significance, which means there is no sufficient evidence that the visual reaction time tends to be slower than the auditory reaction time.
\item
With a Wilcoxon signed rank test, we hence test $H_0 : M\geqslant M_0=0$, calculate
$$W_+=1.5+9.5+3+4.5+7.5+6=32,$$
$$|W_-|=1.5+11+7.5+13+12+15+4.5+9.5+14=88.$$
We take $W=W_+=32$, and with $n = 15$ observations we have the critical value of 30 for a one-tailed test with $P = 0.05$, since $W=32>30$, we failed to reject $H_0$ a 5\% level of significance, which means there is no sufficient evidence that the visual reaction time tends to be slower than the auditory reaction time.

\end{enumerate}

\subsection{}
\begin{enumerate}[label=\roman*)]
\item
We are using Fisher’s Hypothesis Test, we test $$H_0:p_1\leqslant p_2,$$
where $p_1$ denotes the proportion of successful repairs of larger tears and $p_2$ denotes that of shorter tears. Estimators for $p_1$ and $p_2$ are $$\hat{p_1}=14/18=\frac{7}{9}\quad,\quad\hat{p_2}=22/30=\frac{11}{15}.$$
The pooled estimate for the common population proportion is
$$\hat{p}=\frac{n_1\hat{p_1}+n_2\hat{p_2}}{n_1+n_2}=\frac{14+22}{18+30}=\frac{3}{4}.$$
The observed value of the test statistic is
$$Z=\frac{\hat{p_1}-\hat{p_2}}{\sqrt{\hat{p}(1-\hat{p})\left(\frac{1}{n_1}+\frac{1}{n_2}\right)}}=\frac{\frac{7}{9}-\frac{11}{15}}{\sqrt{\frac{3}{4}}\cdot\frac{1}{4}\cdot\left(\frac{1}{18}+\frac{1}{30}\right)}=\frac{4\sqrt{15}}{45}\approx0.344.$$
From the standard normal table, we see that the probability of observing
this large or a larger value is 0.36317, so we shall not reject $H_0$. Therefore, there is no sufficient evidence that the success rate is greater for longer tears.
\item
$$(\hat{p_1}-\hat{p_2})-z_{\alpha}\sqrt{\frac{\hat{p_1}(1-\hat{p_1})}{n_1}-\frac{\hat{p_2}(1-\hat{p_2})}{n_2}}=\left(\frac{7}{9}-\frac{11}{15}\right)-1.645\cdot\sqrt{\frac{\frac{7}{9}\cdot\frac{2}{9}}{18}+\frac{\frac{11}{15}\cdot\frac{4}{15}}{30}}\approx-0.1533.$$
So a 95\% lower confidence bound for $p_1-p_2$ is $p_1-p_2\geqslant -0.1533$, and since $-0.1533<0$, there is no sufficient evidence that the success rate is greater for longer tears.
\end{enumerate}

\subsection{}
The density function of binomial distribution is
$$f_X(x)=\binom{n}{x}p^x(1-p)^{n-x}.$$
Obtain a random sample $x_1,\cdots,x_m$ from the distribution, a maximum likelihood estimator for $p$ is
$$L(p)=\prod_{i=1}^n f_X(x_i)=\prod_{i=1}^m\binom{n}{x_i}p^{x_i}(1-p)^{n-{x_i}}.$$
We take the logarithm of the above expression:
$$\ln L(p)=\sum_{i=1}^m\left[\ln\binom{n}{x_i}+x_i\ln p+(n-x_i)\ln(1-p)\right]=\sum_{i=1}^m\ln\binom{n}{x_i}+\sum_{i=1}^mx_i\ln p+\left(mn-\sum_{i=1}^mx_i\right)\ln(1-p).$$
Maximizing $ln L(p)$ will also maximize $L(p)$, so we take the first derivative and set it equal to zero:
$$\frac{d\ln L(p)}{dp}=\frac{1}{p}\sum_{i=1}^mx_i-\frac{1}{1-p}\left(mn-\sum_{i=1}^mx_i\right)=0,$$
$$p=\frac{1}{mn}\sum_{i=1}^mx_i=\frac{\overline{x}}{n}.$$
According to the data provided, we calculate a maximum-likelihood estimator for $p$ is
$$\hat{p}=\frac{\overline{x}}{n}=\frac{0\cdot39+1\cdot23+2\cdot12+3\cdot1}{75\cdot24}=\frac{1}{36}.$$
In order to apply the binomial distribution, we first calculate
\begin{align*}
P[X=0]=\binom{24}{0}\left(\frac{1}{36}\right)^0\left(\frac{35}{36}\right)^{24}\approx0.5086,\quad & E_0=75\cdot0.5086\approx38.145,\\
P[X=1]=\binom{24}{1}\left(\frac{1}{36}\right)^1\left(\frac{35}{36}\right)^{23}\approx0.3488,\quad & E_1=75\cdot0.3488\approx26.156,\\
P[X=2]=\binom{24}{2}\left(\frac{1}{36}\right)^2\left(\frac{35}{36}\right)^{22}\approx0.1146,\quad & E_2=75\cdot0.1146\approx8.594,\\
P[X\geqslant3]=1-P[X=0]-P[X=1]-P[X=2]\approx0.028,\quad & E_3=75\cdot0.028\approx2.1.
\end{align*}
Since $E_3<5$, a Pearson Statistic is not satisfied, so the problem can be solved by combining the last two categories, making $E_2'=E_2+E_3=10.694$, $P[X\geqslant 2]=0.1426$. The test
$$H_0: {\rm the\ number\ of\ underfilled\ bottles\ follows\ a\ binomial\ distribution\ with\ parameter\ } k=\frac{1}{36}$$
is then equivalent to the test
$$H_0: {\rm the\ number\ of\ underfilled\ bottles\ follows\ a\ categorical\ distribution\ with\ parameter\ } (0.5086,0.3488,0.1426).$$
For $N = 3$ categories, the statistic 
$$X^2=\sum_{i=1}^N\frac{(O_i-E_i)^2}{E_i}$$
then follows a chi-squared distribution with $N-1-m=3-1-1=1$ degree of freedom. We want to realize $\alpha = 0.05$ and therefore reject $H_0$ if $X^2 > \chi^2_{0.05,1} = 3.84$. Now
$$X^2=\frac{(39-38.145)^2}{38.145}+\frac{(23-26.156)^2}{26.156}+\frac{(13-10.694)^2}{10.694}\approx0.897<3.84,$$
so we are unable to reject $H_0$ at the 5\% level of significance, there is no evidence that a binomial distribution is not an appropriate model.

\subsection{}
Let $n_{ij}$ donate number of failure on mounting position $i$ and failure type $j$, and we want to test
$$H_0:{\rm there\ is\ no\ dependence\ between\ mounting\ position\ and\ failure\ type }.$$
We calculate the expected frequencies assuming that $H_0$ is true:
\begin{align*}
&E_{11}=\frac{26\cdot95}{134}\approx18.43,\quad E_{12}=\frac{63\cdot95}{134}\approx44.66,\quad E_{13}=\frac{24\cdot95}{134}\approx17.01,\quad E_{14}=\frac{21\cdot95}{134}\approx14.89,\\
&E_{21}=\frac{26\cdot39}{134}\approx7.57,\quad E_{22}=\frac{63\cdot39}{134}\approx18.34,\quad E_{23}=\frac{24\cdot39}{134}\approx6.99,\quad E_{24}=\frac{21\cdot39}{134}\approx6.11.
\end{align*}
It is now a simple matter to calculate the value of the statistic
$$X^2_{(r-1)(c-1)}=\sum_{i=1}^r\sum_{j=1}^c\frac{(O_{ij}-E_{ij})^2}{E_{ij}},$$
\begin{align*}
X_3^2=&\frac{(22-18.43)^2}{18.43}+\frac{(46-44.66)^2}{44.66}+\frac{(18-17.01)^2}{17.01}+\frac{(9-14.89)^2}{14.89}\\
&+\frac{(4-7.57)^2}{7.57}+\frac{(17-18.34)^2}{18.34}+\frac{(6-6.99)^2}{6.99}+\frac{(12-6.11)^2}{6.11}\\
\approx &0.69+0.04+0.06+2.33+1.68+0.10+0.14+5.68\\
=&10.72.
\end{align*}
Since the statistic follows a chi-squared distribution with 3 degrees of
freedom and if we set $\alpha = 0.05$, we compare this value with $\chi^2_{0.05,3}=7.81$. As $10.72>7.81$, we may reject $H_0$ at 5\% level of significance. There is evidence that the type of failure is dependent of the mounting position.

\subsection{}
Let $n_{ij}$ donate number of salary increase on gender $i$ and increase rate $j$, and we want to test
$$H_0:{\rm there\ is\ no\ dependence\ between\ gender\ and\ increase\ rate }.$$
We calculate the expected frequencies assuming that $H_0$ is true:
\begin{align*}
&E_{11}=71\cdot\frac{2}{3}=\frac{142}{3},E_{12}=74\cdot\frac{2}{3}=\frac{148}{3},E_{13}=153\cdot\frac{2}{3}=102,E_{14}=111\cdot\frac{2}{3}=74,E_{15}=41\cdot\frac{2}{3}=\frac{82}{3},\\
&E_{21}=71\cdot\frac{1}{3}=\frac{71}{3},E_{22}=74\cdot\frac{1}{3}=\frac{74}{3},E_{23}=153\cdot\frac{1}{3}=51,E_{24}=111\cdot\frac{1}{3}=37,E_{25}=41\cdot\frac{1}{3}=\frac{41}{3}.
\end{align*}
It is now a simple matter to calculate the value of the statistic
$$X^2_{(r-1)(c-1)}=\sum_{i=1}^r\sum_{j=1}^c\frac{(O_{ij}-E_{ij})^2}{E_{ij}},$$
\begin{align*}
X_4^2=&\frac{(50-142/3)^2}{142/3}+\frac{(47-148/3)^2}{148/3}+\frac{(103-102)^2}{102}+\frac{(76-74)^2}{74}+\frac{(24-82/3)^2}{82/3}\\
&+\frac{(21-71/3)^2}{71/3}+\frac{(27-74/3)^2}{74/3}+\frac{(50-51)^2}{51}+\frac{(35-37)^2}{37}+\frac{(17-41/3)^2}{41/3}\\
\approx &0.15+0.11+0.01+0.05+0.41+0.30+0.22+0.02+0.11+0.81\\
=&2.19.
\end{align*}
Since the statistic follows a chi-squared distribution with 4 degrees of
freedom and $\chi^2_{0.7,3}\approx2.19$. So the P-value of the test is 0.7 (very large). These data don't tend to support the claim that there is an association between the percentage increase in the
salary of the worker and the worker’s gender. \medskip

In a practical sense, by inspecting the data of the above table, we can find that the percentages of numbers of male and female are very similar in the interval 2-5\%, 6-9\% and 10-13\%, and there are no apparent differences in the interval $<$2\% and $>$14\%, so there is also no evidence of the association.

\end{document}


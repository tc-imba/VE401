\documentclass[conf]{new-aiaa}

\usepackage{amsmath,amssymb}
\usepackage{array}
\usepackage{float}
\usepackage{graphicx}
\usepackage{geometry}


\usepackage{float}
\usepackage{diagbox}
\usepackage[table,xcdraw]{xcolor}

\usepackage{siunitx}
\usepackage{longtable,tabularx}
\usepackage{caption}
\usepackage{subcaption}

\usepackage{minted}
\usemintedstyle{autumn}
\setminted{linenos,breaklines,tabsize=4,xleftmargin=1.5em}

\title{VE401 Term Project 2\\
Rules of metrological testing for net quantity of products in prepackage with fixed content}
\author{Group 21\\
Yihao Liu (515370910207)
\footnote{Authors' names are in alphabet order.}
}
\affil{University OF Michigan - Shanghai Jiao Tong University Joint Institute, Shanghai, China}

\begin{document}

\maketitle

\begin{abstract}
    
\end{abstract}

\vspace{11cm}
\noindent Keywords: 

\newpage

\tableofcontents
\newpage


\section{Nomenclature}

{\renewcommand\arraystretch{1.0}
\noindent\begin{longtable*}{@{}l @{\quad=\quad} l@{}}
$N$ & total number of inspection lot
\end{longtable*}}

\newpage

\section{Problem Statement}

\subsection{Introduction}



\subsection{Objectives}



\subsection{Definitions}



\newpage

\section{Description of work}

\subsection{q6}

Suppose that the number of fatal police shoot between January and March 2019 follows a Poisson distribution with unknown parameter $k$. We want
to determine if there is evidence that this claim is false.

According to the data obtained from xxx, there are 90 days in this three months, so that the random sample size is $N=90$.

\begin{tabular}{cc}
Number & Frequency \\\hline
0 & 8 \\
1 & 16 \\
2 & 22 \\
3 & 18 \\
4 & 11 \\
5 & 7 \\
6 & 5 \\
7 & 2 \\
8 & 0 \\
9 & 1 
\end{tabular}

(according to some proofs in q2 copied from example 2.2.7)

We know that a maximum-likelihood estimator for $k$ is the sample mean,
$$\hat{k}=\overline{X}=\frac{1}{90}
(0\cdot8+1\cdot16+2\cdot22+3\cdot18+4\cdot11
+5\cdot7+6\cdot5+7\cdot2+9\cdot1)=\frac{41}{15}.$$

In order to apply the multinomial distribution, we first calculate
\begin{align*}
P[X=0]&=\frac{e^{-\hat{k}}\hat{k}^0}{0!}=0.0650023\\
P[X=1]&=\frac{e^{-\hat{k}}\hat{k}^1}{1!}=0.177673\\
P[X=2]&=\frac{e^{-\hat{k}}\hat{k}^2}{2!}=0.24282\\
P[X=3]&=\frac{e^{-\hat{k}}\hat{k}^3}{3!}=0.221236\\
P[X=4]&=\frac{e^{-\hat{k}}\hat{k}^4}{4!}=0.151178\\
P[X=5]&=\frac{e^{-\hat{k}}\hat{k}^5}{5!}=0.0826438\\
P[X=6]&=\frac{e^{-\hat{k}}\hat{k}^6}{6!}=0.0376488\\
P[X\geqslant7]&=1-\sum_{i=0}^6P[X=i]=0.0217996
\end{align*}

expected frequencies:
5.8502, 15.9906, 21.8538, 19.9112, 13.606, 7.43794, 3.38839, 1.96196

Last two added and get 5.35036.

$$X^2=\sum_{i=1}^N\frac{(O_i-E_i)^2}{E_i}=2.81152$$
$$N-1-m=7-1-1=5$$
$$\chi^2_{0.72,5}=2.81$$

So the P-value of the test is 0.72, too large, unable to reject Poisson distribution

$$\hat{k}_{2019}=\frac{41}{15}$$

\newpage

\section{Test Results}


\begin{thebibliography}{9}
\bibitem{horst}Horst Hohberger, \emph{VE401,Probabilistic Methods in Eng. Slides}, University OF Michigan - Shanghai Jiao Tong University Joint Institute, Shanghai, China

\end{thebibliography}

\end{document}

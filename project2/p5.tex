\documentclass{article}[12pt]
\usepackage[margin=2.5cm]{geometry}
\usepackage{enumerate}
\usepackage{amsmath}
\usepackage{amssymb}
\usepackage{ulem}
\usepackage{graphicx}
\usepackage{subfigure}
\usepackage{geometry}
\usepackage{multirow}
\usepackage{multicol}
\usepackage{indentfirst}
\usepackage{xcolor}
\usepackage{verbatim}

\begin{document}
Let the ramdom variable $X$ denote the everyday occurrence of police shooting in the United States. We have verified that $X$ follows a Poisson distribution, $p(x)=\frac{(k)^{x}}{x !} e^{k}$. Now we are of interest in the confidence inverval for the parameter $k$. 

We have already known that the a maximum-likelihood estimator for $k$ is the sample mean, $\overline{X}$. 

Since our sample size is sufficient large, with the central limit theorem, it is advisable to say that our sample mean follows a normal distribution. Furthermore, we have a standard normal distribution
\begin{equation}
    Z=\frac{\overline{X}-k}{\sigma / \sqrt{n}} 
\end{equation}


We then need to find and estimator for the variance $\sigma^2$. With the property Poisson distribution. The variance of $X$ should be $Var X = k$. Therefore, the estimator for variance is the estimator for k, which is the sample mean, $\overline{X}$.

Now we rewrite the equ.1 as 
$$
Z = \frac{\hat{k}-k}{ \hat{k}/ \sqrt{n}}
$$
where $\hat{k}$ is the estimator for $k$, which is $\overline{X}$.

It is obvious that the a $(1 - \alpha)100\%$ confidence inverval is 
$$
\hat{k} \pm z_{\alpha / 2} \sqrt{\hat{k} / n}
$$

We can find that there were 3943 police shootings of the year 2015 to 2018. We can calclulate the mean occurrence of one day by
$$
\overline{X} = \frac{3943}{365+366+365+365} = 2.70 
$$
The 95\% confidence inverval can be calclulated as 
$$
2.70 \pm 1.96 \sqrt{2.7/3943} = (2.65 , 2.75)
$$

\end{document}